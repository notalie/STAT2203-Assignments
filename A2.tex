\documentclass[oneside, a4paper]{article}
\usepackage[utf8]{inputenc}
\usepackage[english]{babel}
\usepackage[hypertexnames=false]{hyperref} 
\hypersetup{
    colorlinks=true,
    linkcolor=black,
    filecolor=magenta,      
    urlcolor=cyan,
}

\urlstyle{same}
\usepackage{textcomp}
\usepackage[utf8]{inputenc}
\usepackage{graphicx}
\usepackage{array}
\usepackage{soul}
\usepackage{forest}
\usepackage{mathtools}
\DeclarePairedDelimiter\ceil{\lceil}{\rceil}
\DeclarePairedDelimiter\floor{\lfloor}{\rfloor}
\usepackage{amssymb}

% Set spacing (i set it to 1.2x)
\renewcommand{\baselinestretch}{1}
% Indentation (set this to zero for normal prose)
\setlength{\parindent}{0em}
% Line breaking (spacing between paragraphs)
\setlength{\parskip}{0.5em}

% Use the whole page
\usepackage{geometry}
% Extra math glyphs
\usepackage{amsmath}
% Proper enumerate spacing
\usepackage{enumitem}
% More pleasing screen fonts
\usepackage{lmodern}
% Fancy headers
\usepackage{fancyhdr}
\usepackage{graphicx}
\usepackage{algpseudocode}
% Allows absolute positioning of images
\usepackage{float}
% \usepackage[section]{placeins}
% Set no separation
\setlist{noitemsep}
% Set margins to reasonable
\geometry{margin=2.5cm}
% Sets graphics path
\graphicspath{ {./images/} }
% Sets up fancy headers

\addto\captionsenglish{
}

\usepackage{listings}
\usepackage{color}

\pagestyle{plain}
\NewEnviron{myequation}{%
\begin{equation}
\scalebox{1.5}{$\BODY$}
\end{equation}
}

\begin{document}

\definecolor{dkgreen}{rgb}{0,0.6,0}
\definecolor{gray}{rgb}{0.5,0.5,0.5}
\definecolor{mauve}{rgb}{0.58,0,0.82}

\lstset{frame=tb,
  language=Python,
  aboveskip=3mm,
  belowskip=3mm,
  showstringspaces=false,
  columns=flexible,
  basicstyle={\small\ttfamily},
  numbers=none,
  numberstyle=\tiny\color{gray},
  keywordstyle=\color{blue},
  commentstyle=\color{dkgreen},
  stringstyle=\color{mauve},
  breaklines=true,
  breakatwhitespace=true,
  tabsize=3
}

\pagestyle{fancy}
\fancyhf{}
\lhead{s4530974 - Homework Assignment 1}
\rhead{STAT2203}

\newpage

\setcounter{secnumdepth}{-1}
\section{Question 1}

\textbf{1a)} 

\begin{equation*}
    \begin{split}
       \overline{X} \pm z_{1 - \alpha / 2} \frac{\sigma}{\sqrt{n}} \\ 
       \overline{X} = 0.21 \\
       \sigma = 0.27 \\
       n = 15 \\
       \text{Solve for Alpha} \\
       0.95 = 1 - \alpha, \alpha = 0.05 \\
       \text{Solve for Z Score} \\
       z_{1 - \alpha /2; n - 1} = z_{1 - 0.05 / 2; 15 - 1} \\
        z_{0.975; 14} = 2.145 \\
        \text{Sub into the Equation} \\
       = 0.21 \pm z_{1-0.05/2} \frac{0.27}{\sqrt{15}} \\ 
       = 0.21 \pm 2.145 \frac{0.27}{\sqrt{15}} \\
       = 0.21 \pm 0.14953
    \end{split}
\end{equation*}

The confidence interval range is between 0.06047 and 0.35953.

\textbf{1b)}

\begin{equation*}
    \begin{split}
        \text{Test statistic } = \frac{\overline{X} - \overline{Y}}{\sqrt{\frac{\sigma_x^2}{n_1} + \frac{\sigma_y^2}{n_2}}} \\\\
        \overline{X} = 0.21, \overline{Y} = 0.01, n_1 = 15, n_2 = 17, \sigma_x = 0.27, \sigma_y = 0.24 \\\\
        = \frac{0.21 - 0.01}{\sqrt{\frac{0.27^2}{15} + \frac{0.24^2}{17}}} \\\\
        \text{Test statistic } = 2.20216
    \end{split}
\end{equation*}

\subsection{Finding the P-value and null and alternative hypothesis}
The null hypothesis states that drug administration and placebo are independent while the alternative hypothesis states that there is association between drug administration and placebo.

Using the test statistic, we can calculate the p-value to be $0.01 \leq 0.014 \leq 0.05$. Therefore we can conclude this as moderate evidence against the null hypothesis suggesting an association between drug administration and placebo.


\newpage

\section{Question 2}
\textbf{2a)}

\begin{equation*}
    \begin{split}
        \text{Calculate Alpha} \\
        1 - \alpha = 0.99 \\
        0.01 = \alpha / 2 = 0.005 \\
        125 - 1 = \text{ degrees of freedom.} \\
        z_{0.995;124} =  2.6157 \\
        \overline{X} \pm z_{1-\alpha2}\sqrt{\frac{P(1-P)}{n}}  \\
        \overline{X} = \frac{69}{125} = 0.552 \\
        0.552 \pm 2.6157 \sqrt{\frac{0.552(1-0.552)}{125}} \\
        = 0.552 \pm 0.11634
    \end{split}
\end{equation*}

Therefore the confidence interval for the mean percentage for the proportion of Queenslanders that participated in sports during the time period is between 0.43566 and 0.66834.

\textbf{2b)}

\subsection{The null and alternative hypothesis}

The null hypothesis states that there is no association between states and their participation in sports

The alternative hypothesis states that there may be an association between states and their participation in sports


\subsection{Calculating the test statistic}

\begin{equation*}
    \begin{split}
        P_1 = \frac{69}{125} = 0.552, P_2 = \frac{56}{84} = 0.66 \\
        n_1 = 125, n_2 = 84 \\
        P = \frac{69+56}{125+84} = 0.59808 \\ \\
        \text{Test Statistic } = \frac{P_1 - P_2}{\sqrt{P(1-P)} \sqrt{\frac{1}{n_1}+{\frac{1}{n_2}}}} \\ \\
        = \frac{0.552 - 0.66}{\sqrt{0.59(1-0.59)} \sqrt{\frac{1}{125}+{\frac{1}{84}}}} \\
        \text{Test Statistic } = -1.65771
    \end{split}
\end{equation*}

P-value = 2 * CDF 

The CDF of the Test Statistic is 0.04868902 so the P-value is 0.09737804. This suggests that there is weak evidence against the null hypothesis suggesting there may not be an association between states and their their participation in sports.


\newpage
\section{Question 3}

\begin{center}
    Using the supplied contingency table:

    \begin{table}[H]
        \centering
        \begin{tabular}{|l|l|l|l|}
        \hline
        \textbf{Pressed for Time} & \textbf{Males} & \textbf{Females} & \textbf{Total} \\ \hline
        Always/Often              & 36             & 45               & 81             \\ \hline
        Sometimes                 & 29             & 35               & 64             \\ \hline
        Rarely/Never              & 36             & 19               & 55             \\ \hline
        \textbf{Total}            & 101            & 99               & 200            \\ \hline
        \end{tabular}
    \end{table}

    Multiplying the totals together:
    \begin{table}[H]
        \centering
        \begin{tabular}{|l|l|l|}
        \hline
        \textbf{Pressed for Time} & \textbf{Males} & \textbf{Females} \\ \hline
        \\[-1em]
        Always/Often              & $\frac{81*101}{200}$              & $\frac{81*99}{200}$               \\ \hline
        \\[-1em]
        Sometimes                 & $\frac{64*101}{200}$             & $\frac{64*99}{200}$               \\ \hline
        \\[-1em]
        Rarely/Never              & $\frac{55*101}{200}$             & $\frac{55*99}{200}$               \\ \hline
        \end{tabular}
    \end{table}
    
    Final table created:
    \begin{table}[H]
        \centering
        \begin{tabular}{|l|l|l|}
        \hline
        \textbf{Pressed for Time} & \textbf{Males} & \textbf{Females} \\ \hline
        \\[-1em]
        Always/Often              & $\frac{8181}{200}$              & $\frac{8019}{200}$               \\ \hline
        \\[-1em]
        Sometimes                 & $\frac{808}{25}$             & $\frac{792}{25}$               \\ \hline
        \\[-1em]
        Rarely/Never              & $\frac{1111}{40}$             & $\frac{1089}{40}$               \\ \hline
        \end{tabular}
    \end{table}
\end{center}

\begin{equation*}
    \begin{split}
        X^2 = \frac{(36-\frac{8181}{200})^2}{\frac{8181}{200}} + \frac{(45-\frac{8019}{200})^2}{\frac{8019}{200}} + \frac{(29-\frac{808}{25})^2}{\frac{808}{25}} + \frac{(35-\frac{792}{25})^2}{\frac{792}{25}} + \frac{(36-\frac{1111}{40})^2}{\frac{1111}{40}} + \frac{(19-\frac{1089}{40})^2}{\frac{1089}{40}} \\
    \end{split}
\end{equation*}
$$ = \frac{83075}{12221}$$
$$ X^2 = 6.7977252271$$

\subsection{The Null and Alternative Hypothesis}

The null hypothesis states that there is no association between Gender and 'Feeling rushed or pressed for time'

The alternative hypothesis states that there is an association between Gender and 'Feeling rushed or pressed for time'

\subsection{Calculating the P-Value}

P-value - $P(X^2_{(r-1)(c-1)} \geq X^2), r = 3, c = 2$

Find $P(X^2_{2} \geq 6.7977)$

$P(X^2_2 \geq 5.991) = 0.05$ and $P(X^2_2 \leq ) = 0.025$

Therefore the P-value for the table is between 0.05 and 0.025. This P-value means that there is moderate evidence against the null hypothesis, suggesting an associaton between Gender and 'Feeling rushed or pressed for time'.

\newpage
\section{Question 4}

\textbf{4a)}


\textbf{4b)}



\textbf{4c)}



\section{Question 5}

\subsection{Finding the Expected Value of Y}
$E[Y]$ of a Normal distribution $ = \frac{a+b}{2}$ and the values of the normal has $ a = 0, b = 1$. Therefore, the expected value of Y is: $\frac{1}{2}$

\subsection{Finding the Variance of Y}

\begin{center}
    $$\newcommand{\Var}{\mathrm{Var}} \Var(Y) = E[Y^2] - (E[Y])^2$$

    Since we know that $\newcommand{\Var}{\mathrm{Var}} \Var(Y|X) = x^2$ so $E[Y^2|X] = x^2 + x^2 = 2X^2$

    We can then conclude that: $E[Y^2] = E[E[Y|X]] = E[2X^2]$

    $$ \newcommand{\Var}{\mathrm{Var}} \Var(X) = \frac{(b-a)^2}{12} = \frac{1}{12} , E[X] = \frac{1}{2}$$

    Rearranging we get
    $$ \newcommand{\Var}{\mathrm{Var}} E[V^2] = \Var(V) + (E[U])^2$$

    $$\newcommand{\Var}{\mathrm{Var}} E[2X^2] = 2\left(\Var(X) + E[X]^2\right)$$
    $$ = 2\left(\frac{1}{12} + \left(\frac{1}{2}\right)^2\right)$$
    $$ = \frac{2}{3} = E[Y^2]$$

    Sub into the Variance Equation
    $$ \newcommand{\Var}{\mathrm{Var}} \Var(Y) = E[Y^2] - (E[Y])^2 = \frac{2}{3} - \left(\frac{1}{2}\right)^2$$
    $$ = \frac{2}{3} - \frac{1}{4} $$
    $$ \newcommand{\Var}{\mathrm{Var}} \Var(Y) = \frac{5}{12}$$
\end{center}

\end{document}