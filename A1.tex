\documentclass[oneside, a4paper]{article}
\usepackage[utf8]{inputenc}
\usepackage[english]{babel}
\usepackage[hypertexnames=false]{hyperref} 
\hypersetup{
    colorlinks=true,
    linkcolor=black,
    filecolor=magenta,      
    urlcolor=cyan,
}

\urlstyle{same}
\usepackage{textcomp}
\usepackage[utf8]{inputenc}
\usepackage{graphicx}
\usepackage{array}
\usepackage{soul}
\usepackage{forest}
\usepackage{mathtools}
\DeclarePairedDelimiter\ceil{\lceil}{\rceil}
\DeclarePairedDelimiter\floor{\lfloor}{\rfloor}
\usepackage{amssymb}

% Set spacing (i set it to 1.2x)
\renewcommand{\baselinestretch}{1}
% Indentation (set this to zero for normal prose)
\setlength{\parindent}{0em}
% Line breaking (spacing between paragraphs)
\setlength{\parskip}{0.5em}

% Use the whole page
\usepackage{geometry}
% Extra math glyphs
\usepackage{amsmath}
% Proper enumerate spacing
\usepackage{enumitem}
% More pleasing screen fonts
\usepackage{lmodern}
% Fancy headers
\usepackage{fancyhdr}
\usepackage{graphicx}
\usepackage{algpseudocode}
% Allows absolute positioning of images
\usepackage{float}
% \usepackage[section]{placeins}
% Set no separation
\setlist{noitemsep}
% Set margins to reasonable
\geometry{margin=2.5cm}
% Sets graphics path
\graphicspath{ {./images/} }
% Sets up fancy headers

\addto\captionsenglish{
}


\usepackage{listings}
\usepackage{color}

\pagestyle{plain}
\NewEnviron{myequation}{%
\begin{equation}
\scalebox{1.5}{$\BODY$}
\end{equation}
}

\begin{document}

\definecolor{dkgreen}{rgb}{0,0.6,0}
\definecolor{gray}{rgb}{0.5,0.5,0.5}
\definecolor{mauve}{rgb}{0.58,0,0.82}

\lstset{frame=tb,
  language=Python,
  aboveskip=3mm,
  belowskip=3mm,
  showstringspaces=false,
  columns=flexible,
  basicstyle={\small\ttfamily},
  numbers=none,
  numberstyle=\tiny\color{gray},
  keywordstyle=\color{blue},
  commentstyle=\color{dkgreen},
  stringstyle=\color{mauve},
  breaklines=true,
  breakatwhitespace=true,
  tabsize=3
}

\pagestyle{fancy}
\fancyhf{}
\lhead{s4530974 - Homework Assignment 1}
\rhead{STAT2203}

\newpage

\setcounter{secnumdepth}{-1}
\section{Question 1}

\textbf{1a)} The probability of $F_Y(y) = P(Y \leq y)$ is and since $Y = log(X - 1)$. 

The rule $ln(N) = x \equiv N = e^x$ is also important to note.

\begin{equation}
\begin{split}
F_Y(y) = P(Y\leq y) \\
= P(log(X-1)\leq y) \\
= P(log(X-1)\leq y) \\
= P(X - 1\leq e^y) \\
= P(X \leq e^y + 1) \\
\implies 
F_Y(y) = P(X \leq e^y + 1)
\end{split}
\end{equation}
To find the function of Y ($f_y(y)$), integrate by the bounds of the X which are $e^y + 1$ and 2.

\begin{equation}
\begin{split}
= \int_{2}^{e^y + 1} \frac{2}{x^2} dx \\
= \left[  -\frac{2}{x} \right]^{e^y + 1}_2 \\
= -\frac{2}{1 - e^y} + \frac{2}{2} \\
= -\frac{2}{1 - e^y} + 1 \\
\implies F_Y(y) = -\frac{2}{1 - e^y} + 1
\end{split}
\end{equation}

Since we now have the function of Y, we need to find the pdf of Y by differentiating $F_Y(y)$.

\begin{equation}
    \begin{split}
    \frac{\partial }{\partial y} F_Y(y) = \frac{\partial }{\partial y}-\frac{2}{1 - e^y} + 1 \\
    \left(v = (1 - e^y), v' = -e^y, u = 2, u' = 0  \right)\\
    = -\frac{2e^y}{\left(1 -e^y \right)^2} \\\\
    \implies f_y(y) = -\frac{2e^y}{\left(1 -e^y \right)^2}, x > 2 \\\\
        f_y(y)= 
    \begin{cases}
        = -\frac{2e^y}{\left(1 -e^y \right)^2},& \text{if } y\geq 2\\
        0,              & \text{otherwise}
    \end{cases}
    \end{split}
\end{equation}

\textbf{1b)} To find the quartile function. Find the inverse of the cdf. To do this, switch x and q and solve for q.

\begin{equation}
    \begin{split}
    F_Y(y) = -\frac{2}{1 - e^y} + 1 = q \\
    x = -\frac{2}{1 - e^q} + 1 \\
    x - 1 = -\frac{2}{1 - e^q} \\ 
    1 - e^q = -\frac{2}{x - 1} \\ 
    e^q - 1 = \frac{2}{x - 1} \\ 
    e^q = \frac{2}{x - 1} + 1 \\ 
    q = \ln{\left(\frac{2}{x - 1} + \frac{x - 1}{x - 1}\right)} \\ 
    q = \ln{\left(\frac{x + 1}{x - 1}\right)} \\ 
    \implies qy(y) = \ln{\left(\frac{x + 1}{x - 1}\right)}
    \end{split}
\end{equation} 
\textbf{1c)} Randomised code for the pdf function of $f_X(x)$:
    \begin{lstlisting}[language=R]
        n = 250
        u = runif(n)
        X = log(abs((u+1)/(u - 1)))
        hist(X,freq=FALSE)
        x = seq(from=0,to=1,length=250)
        lines(x, 2/x^2,lwd=2)
    \end{lstlisting}

\section{Question 2}

\textbf{2a)} Expression that the system is working at time $t$:\\
- Probability density function of components combined can be expressed as a Poisson distribution: $\frac{\lambda ^x}{x!} e^{- \lambda}$ \\
- Since $\frac{1}{\lambda}$ = mean and mean = 1, $1 = \frac{1}{\lambda}$, $1 = \frac{1}{1}, \lambda = 1$ \\ 
- Since the pdf of the Poisson function does not factor in time, we multiply all $\lambda$ by $t$ to get the following function:
\begin{equation}
    \begin{split}
        F_X(x) = \frac{( \lambda t )^x}{x!} e^{- \lambda t}, \lambda = 1\\ 
        F_X(x \geq 2) = \frac{t^x}{x!} e^{-t}, \\
        F_X(2) = \frac{t^2}{2!} e^{-t} \\
        F_X(3) = \frac{t^3}{3!} e^{-t} \\
        F_X(2) + F_X(3) = \frac{t^2}{2!} e^{-t} + \frac{t^3}{3!} e^{-t} \\
        F_X(x \geq 2) = \frac{t^2}{2!} e^{-t} + \frac{t^3}{3!} e^{-t} 
    \end{split}
\end{equation}
which is the function for time to failure for the system.

\textbf{2b)} Since we know that $F_X(x \geq 2) = \frac{t^2}{2!} e^{-t} + \frac{t^3}{3!} e^{-t}$ \\
- We can find the mean value of the function with $\int _{-\infty}^{\infty} t F_X(x \geq 2)$ \\
- And find the $EX^2$ with $\int _{-\infty}^{\infty} t^2 F_X(x \geq 2)$ \\ 

First find $\int _{-\infty}^{\infty} t \left(\frac{t^2}{2!} e^{-t} + \frac{t^3}{3!} e^{-t}\right)$ 
- Since the upper bounds are defined as 3 with lower bound 2 ($X \geq 2$), we can sub these into the integral bounds.
\begin{equation}
    \begin{split}
        \int _{2}^{3} t \left( \frac{t^2}{2!} e^{-t} + \frac{(t)^3}{3!} e^{-t} \right) \\ \\
        \int _{2}^{3} \frac{t^3}{2!} e^{-t} + \frac{t^4}{3!} e^{-t} \\ \\ 
        \int _{2}^{3} \frac{t^3 e^{-t}}{2} + \frac{t^4 e^{-t}}{6} \\ \\ 
        \int _{2}^{3} \frac{t^3 e^{-t}}{2} + \int _{2}^{3} \frac{t^4 e^{-t}}{6} \\ \\ 
        \frac{1}{2} \int _{2}^{3} t^3 e^{-t} + \frac{1}{6} \int _{2}^{3} t^4 e^{-t}
    \end{split}
\end{equation}
Solving the first integral:
\begin{equation}
    \begin{split}
        \frac{1}{2} \int _{2}^{3} t^3 e^{-t} \\ 
        u = -t \\
        \frac{1}{2} \int e^u u^3du \\ 
        u = u^3, v' = e^u\\
        u^3 e^u - \int 3 u^2 e^u du \\ 
        u = -t \\
        \left(\frac{1}{2}\left(-t\right)^3e^{-t}-3\left(\left(-t\right)^2e^{-t}-2\left(e^{-t}\left(-t\right)-e^{-t}\right)\right)\right)\\
        \left[\left(\frac{1}{2}\left(-t\right)^3e^{-t}-3\left(\left(-t\right)^2e^{-t}-2\left(e^{-t}\left(-t\right)-e^{-t}\right)\right)\right) \right]^3_2 \\
        = \frac{1}{2} \left()\frac{38}{e^2} - \frac{78}{e^3}\right) \\
        = -\frac{-38e+78}{2e^3}
    \end{split}
\end{equation}
Solving the second integral:
\begin{equation}
    \begin{split}
        \frac{1}{6} \int _{2}^{3} t^4 e^{-t} \\
        u=t^4, v'=e^{-t} \\
        \frac{1}{6} \left(-e^{-t}t^4 - \int -4e^{-t}t^3dt\right) \\
        \frac{1}{6} \left(-e^{-t}t^4 + 4\left(\left(-t\right)^3e^{-t}-3\left(\left(-t\right)^2e^{-t}-2\left(e^{-t}\left(-t\right)-e^{-t}\right)\right)\right)\right) \\
        \left[-e^{-t}t^4+4\left(-e^{-t}t^3-3\left(e^{-t}t^2-2\left(-e^{-t}t-e^{-t}\right)\right)\right)\right]^3_2 \\
         = \frac{1}{6}\left(-\frac{393}{e^3}+\frac{168}{e^2}\right)
         = \frac{-393+168e}{6e^3}
    \end{split}
\end{equation}
Adding both together to get the mean
\begin{equation}
-\frac{-38e+78}{2e^3}+\frac{-393+168e}{6e^3} = \frac{282e-627}{6e^3}
\end{equation}
Mean  = $\frac{282e-627}{6e^3} = 1.158$

- Variance = $E[X^2] - (E[X])^2$ \\ \\
To find the variance, a similar process was taken to find $E[X^2]$ so that the variance could be \\
calculated. This formula was that of $\int _{-\infty}^{\infty} t^2 \left(\frac{t^2}{2!} e^{-t} + \frac{t^3}{3!} e^{-t}\right)$ and gave the result: \\
$E[X^2] = \frac{1376e-3387}{6e^3}$. It is also worth noting that $E[X]$ is equivilent to the mean or expected value that was just calculated.

The variance could then be calculated which looked like the following:

\begin{equation}
    \frac{1376e-3387}{6e^3} - (\frac{282e-627}{6e^3})^2 \\
    = 1.591
\end{equation}
Var($F_X(x)$)= 1.591

\textbf{1c)} 


\end{document}