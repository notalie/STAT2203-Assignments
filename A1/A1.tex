\documentclass[oneside, a4paper]{article}
\usepackage[utf8]{inputenc}
\usepackage[english]{babel}
\usepackage[hypertexnames=false]{hyperref} 
\hypersetup{
    colorlinks=true,
    linkcolor=black,
    filecolor=magenta,      
    urlcolor=cyan,
}

\urlstyle{same}
\usepackage{textcomp}
\usepackage[utf8]{inputenc}
\usepackage{graphicx}
\usepackage{array}
\usepackage{soul}
\usepackage{forest}
\usepackage{mathtools}
\DeclarePairedDelimiter\ceil{\lceil}{\rceil}
\DeclarePairedDelimiter\floor{\lfloor}{\rfloor}
\usepackage{amssymb}

% Set spacing (i set it to 1.2x)
\renewcommand{\baselinestretch}{1}
% Indentation (set this to zero for normal prose)
\setlength{\parindent}{0em}
% Line breaking (spacing between paragraphs)
\setlength{\parskip}{0.5em}

% Use the whole page
\usepackage{geometry}
% Extra math glyphs
\usepackage{amsmath}
% Proper enumerate spacing
\usepackage{enumitem}
% More pleasing screen fonts
\usepackage{lmodern}
% Fancy headers
\usepackage{fancyhdr}
\usepackage{graphicx}
\usepackage{algpseudocode}
% Allows absolute positioning of images
\usepackage{float}
% \usepackage[section]{placeins}
% Set no separation
\setlist{noitemsep}
% Set margins to reasonable
\geometry{margin=2.5cm}
% Sets graphics path
\graphicspath{ {./images/} }
% Sets up fancy headers

\addto\captionsenglish{
}

\usepackage{listings}
\usepackage{color}

\pagestyle{plain}
\NewEnviron{myequation}{%
\begin{equation}
\scalebox{1.5}{$\BODY$}
\end{equation}
}

\begin{document}

\definecolor{dkgreen}{rgb}{0,0.6,0}
\definecolor{gray}{rgb}{0.5,0.5,0.5}
\definecolor{mauve}{rgb}{0.58,0,0.82}

\lstset{frame=tb,
  language=Python,
  aboveskip=3mm,
  belowskip=3mm,
  showstringspaces=false,
  columns=flexible,
  basicstyle={\small\ttfamily},
  numbers=none,
  numberstyle=\tiny\color{gray},
  keywordstyle=\color{blue},
  commentstyle=\color{dkgreen},
  stringstyle=\color{mauve},
  breaklines=true,
  breakatwhitespace=true,
  tabsize=3
}

\pagestyle{fancy}
\fancyhf{}
\lhead{s4530974 - Homework Assignment 1}
\rhead{STAT2203}

\newpage

\setcounter{secnumdepth}{-1}
\section{Question 1}

\textbf{1a)} The probability of $f_Y(y) = P(Y \leq y)$ is and since $Y = log(X - 1)$. 

The rule $ln(N) = x \equiv N = e^x$ is also important to note.

\begin{equation}
\begin{split}
f_Y(y) = P(Y\leq y) \\
= P(log(X-1)\leq y) \\
= P(log(X-1)\leq y) \\
= P(X - 1\leq e^y) \\
= P(X \leq e^y + 1) \\
\implies 
f_Y(y) = P(X \leq e^y + 1)
\end{split}
\end{equation}

Sub this found X value into the existing x function to get:

$$\frac{2}{(e^y + 1)^2}$$

Then multiply this function by the derivative of X which is $\frac{d}{dy} (e^y + 1) = e^y$ to get:

$$\frac{2e^y}{(e^y + 1)^2}$$

To add bounds to the cases, sub in $x = 2$ into $Y > log(X - 1)$ which gives $Y > ln(1), Y > 0$

\begin{equation}
\begin{split}
    f_y(y) = \begin{cases} 
        \frac{2e^y}{(e^y + 1)^2} & y > 0 \\ 
        0 & otherwise
    \end{cases}
\end{split}
\end{equation}



\textbf{1b)} To find the quartile function. We need to find the we need to integrate the pdf we just got to get Fx(x). 

\begin{equation}
    \begin{split}
    \int^2_x F_X(x) = \int \frac{2}{x^2} \\
    \int^2_x \frac{2}{x^2} = \left[ - \frac{2}{x} \right]_x^2 \\
    = - \frac{2}{x} - -\frac{2}{2} \\
    = 1 - \frac{2}{x} \\
    \text{Switch q and x to get the } \\ 
    x = - \frac{2}{q} + 1 \\
    x - 1 =  - \frac{2}{q} \\
    q (x - 1) = -2 \\
    q = - \frac{2}{x - 1} \\
    \implies qx(x) = - \frac{2}{x - 1} \\
    \end{split}
\end{equation} 
\textbf{1c)} Randomised code for the pdf function of $f_X(x)$:
    \begin{lstlisting}[language=R]
        pdfFunc <- function() {
            n = runif(1, 0, 10)
            if (n <= 2) {
                print(0)
            } else {
                print(2/(n^2))
            }
        }        
    \end{lstlisting}
    
\section{Question 2}

\textbf{2a)} 
- One component = $e^{-t}$ \\
- cdf of the system (failure) $P(T \leq t) = 1- P(T>t)$ \\
- pdf = $\frac{dy}{dx} F_T(t)$

\begin{equation}
    \begin{split}
        P(T>t) = X_1X_2(1-X_3)3 + X_1X_2X_3 \\
        P(T>t) = 3e^{-2t} -2e^{-3t} \\
        P(T \leq t) = 1- P(T>t) = 1 - (3e^{-2t} -2e^{-3t} \\
        P(T \leq t) = 1 - P(T > t) \\
        P(T \leq t) = 1 + 2e^{-3t} - 3e^{-2t} \\
        F_T(t) = 1 + 2e^{-3t} - 3e^{-2t} \\
        f_t(t) = -6e^{-3t} + 6e^{-2t} \\
        \implies \text{pdf of the failure of the system } = -6e^{-3t} + 6e^{-2t}
    \end{split}
\end{equation}
(Scroll down for next part \LaTeX \: hates me)

\textbf{2b)} To find the mean or expected value use the following formula:
$ \int_0^\infty t\left( -6e^{-3t} + 6e^{-2t} \right)$ 

\begin{equation}
    \begin{split}
        \int_0^\infty t\left( -6e^{-3t} + 6e^{-2t} \right)
            = \int_0^\infty -6e^{-3t}t+6e^{-2t}t \\
        -\int \:6e^{-3t}tdt+\int \:6e^{-2t}tdt \\
        \text{Integral of first half} \\
            \int \:6te^{-3t}dt \\
            = 6 \int \:te^{-3t}dt\\
            u = -3t \\
            = 6 \int \frac{e^uu}{9} du \\
            = \frac{6}{9} \int e^uu \: du \\
            u=u,\:v'=e^u \\
            = \frac{6}{9}\left(e^uu-\int \:e^udu\right) 
            = \frac{6}{9}\left(e^uu-e^u\right) \\
            u = -3t \\
            \frac{6}{9}\left(e^{-3t}\left(-3t\right)-e^{-3t}\right) 
            = -2e^{-3t}t-\frac{2}{3}e^{-3t}
        \text{Integral of second half} \\
            \int \:6e^{-2t}tdt \\
            = 6 \int \:e^{-2t}tdt \\
            u=-2t \\
            = 6 \int \frac{e^uu}{4}du \\
            = \frac{6}{4} \int \:e^uudu \\ 
            \:u=u,\:v'=e^u\\
            = \frac{6}{4}\left(e^uu-\int \:e^udu\right) \\
            = \frac{6}{4}\left(e^uu-e^u\right) \\
            u = -2t \\
            = \frac{6}{4}\left(e^{-2t}\left(-2t\right)-e^{-2t}\right) \\
            = -3e^{-2t}t-\frac{3}{2}e^{-2t} \\
        \text{Putting the two integrals together we get: } \\
        \left[2e^{-3t}t+\frac{2}{3}e^{-3t}-3e^{-2t}t-\frac{3}{2}e^{-2t}\right]^\infty_0 \\
        \text{Subbing in the bounds we end up with:} \\
        0-\left(-\frac{5}{6}\right) \\
        \text{Mean: } = \frac{5}{6}
    \end{split}
\end{equation}

I then did something similar to get $E[X^2]$ where I got $\frac{19}{18}$. \\
I then used these values to calculate variance which can be calculated with the formula $E[X^2] - (E[X])^2$

Variance = $\frac{19}{18}\:-\:\left(\frac{5}{6}\right)^2$

Variance = $\frac{13}{36} \approx 0.3611$

\textbf{2c)} To find where $t \to \infty$ find: 
$$\lim_{t \to \infty} \frac{X_1 \cap P(T>t)}{P(T>t)}$$ \\ \\
- We know that the System Working can be given by $P(T>t) = 3e^{-2t} -2e^{-3t}$ \\
- The equation of the System Working and Component 1 Working can be given by $2e^{-2t} - e^{-3t}$ \\

\begin{equation}
    \begin{split}
        \lim_{t \to \infty} \frac{2e^{-2t} - e^{-3t}}{3e^{-2t} -2e^{-3t}} \\
        \text{Divide by the largest denominator factor:}\\
        \lim_{t \to \infty} \frac{2-\frac{1}{e^t}}{3-\frac{2}{e^t}} \\
        \frac{\lim _{t\to \infty }\left(2-\frac{1}{e^t}\right)}{\lim _{t\to \infty }\left(3-\frac{2}{e^t}\right)} \\ 
        \implies \lim _{t\to \infty }\left(2-\frac{1}{e^t}\right)=2 \\
        \implies \lim _{t\to \infty }\left(3-\frac{2}{e^t}\right)=3 \\ 
        \frac{\lim _{t\to \infty }\left(2-\frac{1}{e^t}\right)}{\lim _{t\to \infty }\left(3-\frac{2}{e^t}\right)} = \frac{2}{3} \\ 
        \implies \text{The limiting value as } \lim_{t \to \infty} \frac{2e^{-2t} - e^{-3t}}{3e^{-2t} -2e^{-3t}} = \frac{2}{3}
    \end{split}
\end{equation}

\textbf{2d)} Randomised variables in R were used to generate Exponential values. The code looked like the following: 

\begin{lstlisting}[language=R]
    totalFailTime = c()
    # Runs 10 000 times
    for (i in 1:10000) {
        test<-rexp(3, rate = 1)
        test = sort(test)
        secondComponent = test[2]
        spareComponent = rexp(1, rate = 1)
        lastComponent = test[3]
        totalFailTime[i] <- min(secondComponent + spareComponent, lastComponent)
    }
    
    hist(totalFailTime,freq=FALSE)
    x = seq(from=0,to=1,length=100)
    # Prints the mean
    print(mean(totalFailTime))
\end{lstlisting}

\newpage
\section{Question 3}

\textbf{3a)} The conditional distribution of Y can be expressed as:
$$P(Y = y | X = x)  = \frac{f_{xy}(x,y)}{f_{x}(X = X)}$$

\begin{equation}
    \begin{split}
        P(Y = y | X = x)  = \frac{f_{xy}(x,y)}{f_{x}(X = X)} = Exp(x) \\ \\
        \text{Given P(X = x) = Exp(1)} \\ \\
        P(Y = y | X = x)  = \frac{f_{xy}}{Exp(1)} = Exp(x) \\ \\
        f_{xy}(X, Y) = Exp(x)Exp(1) = e^{-x} \left(xe^{-yx}\right)\\ \\
        \text{Now that we have Fxy(X, Y), we can find the marginal distribution of Y by } F_y(y) = \int^{\infty}_{-\infty} fxy(x,y,) dx \\\\
        f_y(y) = \int^{\infty}_{-\infty} fxy(x,y,) dx \\
        = \int^{\infty}_{-\infty} e^{-x} \left(xe^{-yx}\right) dx \\
        \text{Because -infinity approaches 0, we can change the bounds to that} \\
        = \int^{\infty}_{0} e^{-x} \left(xe^{-yx}\right) dx \\
        = \int e^{-yx-x}xdx \\
        u=x, v'=e^{-yx-x} \\
        = e^{-yx-x}\frac{1}{-y-1}x-\int e^{-yx-x}\frac{1}{-y-1}dx \\
        = e^{-yx-x}\frac{1}{-y-1}x-e^{-yx-x}\frac{1}{\left(-y-1\right)^2} \\
        = \left[e^{-yx-x}\frac{1}{-y-1}x-e^{-yx-x}\frac{1}{\left(-y-1\right)^2}\right]^\infty_0 \\
        = 0 - \left(-\frac{1}{\left(-y-1\right)^2}\right) \\
        F_y(Y) = \frac{1}{\left(-y-1\right)^2}  \\
        \implies \text{The marginal probability density function of Y} = \frac{1}{\left(-y-1\right)^2}
    \end{split}
\end{equation}

\textbf{3b)} Now that we have the marginal pdf of Y, we can calculate $E[X | Y = y]$: \\
$$P(X = x | Y = y)  = \frac{f_{xy}(x,y)}{f_{y}(Y = y)}$$

\begin{equation}
    \begin{split}
        P(X = x | Y = y)  = \frac{f_{xy}(x,y)}{f_{y}(Y = y)}\\
        = \frac{e^{-x} \left(xe^{-yx}\right)}{\frac{1}{\left(-y-1\right)^2}} \\
        = e^{-x-xy}x\left(-y-1\right)^2 \\
        \implies E[X | Y = y] = e^{-x-xy}x\left(-y-1\right)^2
    \end{split}
\end{equation}

\section{Question 4}
- Since we know that 

\begin{equation}
    \begin{split}
        cov(Z) = \begin{bmatrix}
            1 & 0\\
            0 & 1
        \end{bmatrix}, \text{mean} = E(Z) = 0 \\
        \text{Find vector a and B such that:} \\
        Y = a + BZ \text{ where:} \\ \\
        \mu =  \begin{bmatrix}
            3 \\
            -2
        \end{bmatrix}, \Sigma =  \begin{bmatrix}
            5 & 13 \\
            13 & 41
        \end{bmatrix} \\ \\
        E(Y) =  \begin{bmatrix}
            3 \\
            -2 
        \end{bmatrix} \\ \\ 
        E(Y) = a + BE(Z) \\ \\
        \begin{bmatrix}
            3 \\
            -2 
        \end{bmatrix} = a + B \begin{bmatrix}
            0 \\
            0 
        \end{bmatrix} \\ \\ 
        \implies a = \begin{bmatrix}
            3 \\
            -2 
        \end{bmatrix}
    \end{split}
\end{equation}

Now that we have a, we can find B using the covariance


\begin{equation}
    \begin{split}
        cov(Y) = E((Y - E(Y)) (Y - E(Y))) \\
        = E((a + BZ - a) (a + BZ - a)') \\
        = E((BZ) (BZ)') \\
        = BB'E(ZZ') = BB' \begin{bmatrix}
            1 & 0 \\
            0 & 1 
        \end{bmatrix}\\ \\ \\
        cov(Z) = E((Z - E(Z)) (Z - E(Z)') \\
        cov(Z) = E((Z - 0) (Z - 0)') \\
        = E(ZZ') \\
        = \begin{bmatrix}
            1 & 0 \\
            0 & 1 
        \end{bmatrix}\\
        \text{Let B = } 
            \begin{bmatrix}
                a & b \\
                c & d 
            \end{bmatrix}\\
        cov(Y) = 
            \begin{bmatrix}
                a & b \\
                c & d 
            \end{bmatrix} 
            \begin{bmatrix}
                a & b \\
                c & d 
            \end{bmatrix} ' 
            \begin{bmatrix}
                1 & 0 \\
                0 & 1 
            \end{bmatrix} \\ \\
        = 
            \begin{bmatrix}
                a & b \\
                c & d 
            \end{bmatrix} 
            \begin{bmatrix}
                a & b \\
                c & d 
            \end{bmatrix} \\ \\
        = 
            \begin{bmatrix}
                a^2 + b^2 & ac + bd \\
                ca + db & c^2 + d^2 
            \end{bmatrix} \\ \\
        = 
            \begin{bmatrix}
                5 & 13 \\
                13 & 41 
            \end{bmatrix} 
        \implies a^2 + b^2 = 5\\
        ac + bd = 13 \\
        c^2 + d^2 = 41 \\
        \text{Chosen values that will satisfy these equations are: } \\
        a = 0, b = \sqrt{5}, c = \frac{6}{\sqrt{5}}, d = \frac{13}{\sqrt{5}} \\
        \implies B = 
            \begin{bmatrix}
                0 & \sqrt{5} \\
                \frac{6}{\sqrt{5}} & \frac{13}{\sqrt{5}}
            \end{bmatrix} 
    \end{split}
\end{equation}

\end{document}